% Options for packages loaded elsewhere
% Options for packages loaded elsewhere
\PassOptionsToPackage{unicode}{hyperref}
\PassOptionsToPackage{hyphens}{url}
%
\documentclass[
  ignorenonframetext,
  aspectratio=169,
  russian,
]{beamer}
\newif\ifbibliography
\usepackage{pgfpages}
\setbeamertemplate{caption}[numbered]
\setbeamertemplate{caption label separator}{: }
\setbeamercolor{caption name}{fg=normal text.fg}
\beamertemplatenavigationsymbolshorizontal
% remove section numbering
\setbeamertemplate{part page}{
  \centering
  \begin{beamercolorbox}[sep=16pt,center]{part title}
    \usebeamerfont{part title}\insertpart\par
  \end{beamercolorbox}
}
\setbeamertemplate{section page}{
  \centering
  \begin{beamercolorbox}[sep=12pt,center]{section title}
    \usebeamerfont{section title}\insertsection\par
  \end{beamercolorbox}
}
\setbeamertemplate{subsection page}{
  \centering
  \begin{beamercolorbox}[sep=8pt,center]{subsection title}
    \usebeamerfont{subsection title}\insertsubsection\par
  \end{beamercolorbox}
}
% Prevent slide breaks in the middle of a paragraph
\widowpenalties 1 10000
\raggedbottom
\AtBeginPart{
  \frame{\partpage}
}
\AtBeginSection{
  \ifbibliography
  \else
    \frame{\sectionpage}
  \fi
}
\AtBeginSubsection{
  \frame{\subsectionpage}
}
\usepackage{iftex}
\ifPDFTeX
  \usepackage[T1]{fontenc}
  \usepackage[utf8]{inputenc}
  \usepackage{textcomp} % provide euro and other symbols
\else % if luatex or xetex
  \usepackage{unicode-math} % this also loads fontspec
  \defaultfontfeatures{Scale=MatchLowercase}
  \defaultfontfeatures[\rmfamily]{Ligatures=TeX,Scale=1}
\fi
\usepackage{lmodern}

\ifPDFTeX\else
  % xetex/luatex font selection
\fi
% Use upquote if available, for straight quotes in verbatim environments
\IfFileExists{upquote.sty}{\usepackage{upquote}}{}
\IfFileExists{microtype.sty}{% use microtype if available
  \usepackage[]{microtype}
  \UseMicrotypeSet[protrusion]{basicmath} % disable protrusion for tt fonts
}{}


\usepackage{longtable,booktabs,array}
\usepackage{calc} % for calculating minipage widths
\usepackage{caption}
% Make caption package work with longtable
\makeatletter
\def\fnum@table{\tablename~\thetable}
\makeatother
\usepackage{graphicx}
\makeatletter
\newsavebox\pandoc@box
\newcommand*\pandocbounded[1]{% scales image to fit in text height/width
  \sbox\pandoc@box{#1}%
  \Gscale@div\@tempa{\textheight}{\dimexpr\ht\pandoc@box+\dp\pandoc@box\relax}%
  \Gscale@div\@tempb{\linewidth}{\wd\pandoc@box}%
  \ifdim\@tempb\p@<\@tempa\p@\let\@tempa\@tempb\fi% select the smaller of both
  \ifdim\@tempa\p@<\p@\scalebox{\@tempa}{\usebox\pandoc@box}%
  \else\usebox{\pandoc@box}%
  \fi%
}
% Set default figure placement to htbp
\def\fps@figure{htbp}
\makeatother



\ifLuaTeX
\usepackage[bidi=basic,provide=*]{babel}
\else
\usepackage[bidi=default,provide=*]{babel}
\fi
% get rid of language-specific shorthands (see #6817):
\let\LanguageShortHands\languageshorthands
\def\languageshorthands#1{}


\setlength{\emergencystretch}{3em} % prevent overfull lines

\providecommand{\tightlist}{%
  \setlength{\itemsep}{0pt}\setlength{\parskip}{0pt}}



 

\usepackage[]{csquotes}

\IfFileExists{plex-otf.sty}{
  %% Full TeXlive
  % \usepackage[%
  %   % math,
  %   RM={Scale=0.94},SS={Scale=0.94},SScon={Scale=0.94},TT={Scale=MatchLowercase,FakeStretch=0.9},DefaultFeatures={Ligatures=Common}
  % ]{plex-otf}
}{
  %% TinyTeX
  \usepackage{libertine}
}

%%% Load theme
% https://deic.uab.cat/~iblanes/beamer_gallery/
\IfFileExists{beamerthemegotham.sty}{
  %% Full TeXlive
  \usetheme{gotham}
  \gothamset{
    numbering=totalpagenumber,
    parttocframe default=off,
    sectiontocframe default=off,
    subsectiontocframe default=off,
  }
}{
  %% TinyTeX
  \usetheme{Madrid}
}
\makeatletter
\@ifpackageloaded{caption}{}{\usepackage{caption}}
\AtBeginDocument{%
\ifdefined\contentsname
  \renewcommand*\contentsname{Содержание}
\else
  \newcommand\contentsname{Содержание}
\fi
\ifdefined\listfigurename
  \renewcommand*\listfigurename{Список иллюстраций}
\else
  \newcommand\listfigurename{Список иллюстраций}
\fi
\ifdefined\listtablename
  \renewcommand*\listtablename{Список таблиц}
\else
  \newcommand\listtablename{Список таблиц}
\fi
\ifdefined\figurename
  \renewcommand*\figurename{Рисунок}
\else
  \newcommand\figurename{Рисунок}
\fi
\ifdefined\tablename
  \renewcommand*\tablename{Таблица}
\else
  \newcommand\tablename{Таблица}
\fi
}
\@ifpackageloaded{float}{}{\usepackage{float}}
\floatstyle{ruled}
\@ifundefined{c@chapter}{\newfloat{codelisting}{h}{lop}}{\newfloat{codelisting}{h}{lop}[chapter]}
\floatname{codelisting}{Список}
\newcommand*\listoflistings{\listof{codelisting}{Листинги}}
\makeatother
\makeatletter
\makeatother
\makeatletter
\@ifpackageloaded{caption}{}{\usepackage{caption}}
\@ifpackageloaded{subcaption}{}{\usepackage{subcaption}}
\makeatother

\usepackage{bookmark}
\IfFileExists{xurl.sty}{\usepackage{xurl}}{} % add URL line breaks if available
\urlstyle{same}
\hypersetup{
  pdftitle={Лабораторная работа № 1},
  pdfauthor={Перегудов Александр Вадимович},
  pdflang={ru-RU},
  hidelinks,
  pdfcreator={LaTeX via pandoc}}


\title{Лабораторная работа № 1}
\subtitle{Математическое моделирование}
\author{Перегудов Александр Вадимович}
\date{2026-02-18}

\begin{document}
\frame{\titlepage}

\renewcommand*\contentsname{Содержание}
\begin{frame}[allowframebreaks]
  \frametitle{Содержание}
  \setcounter{tocdepth}{1}
  \tableofcontents
\end{frame}
\setcounter{tocdepth}{1}
\tableofcontents
}

\section{Информация}\label{ux438ux43dux444ux43eux440ux43cux430ux446ux438ux44f}

\begin{frame}[fragile]{Докладчик}
\phantomsection\label{ux434ux43eux43aux43bux430ux434ux447ux438ux43a}
\begin{columns}[c]
\begin{column}{0.7\linewidth}
\begin{itemize}[<+->]
\tightlist
\item
  Перегудов Александр Вадимович
\item
  Студент группы НФИбд-02-23
\item
  Российский университет дружбы народов им. П. Лумумбы
\item
  \url{https://github.com/magister6239/mathmod}
\end{itemize}

\section{Вводная
часть}\label{ux432ux432ux43eux434ux43dux430ux44f-ux447ux430ux441ux442ux44c}

\begin{frame}{Актуальность}
\phantomsection\label{ux430ux43aux442ux443ux430ux43bux44cux43dux43eux441ux442ux44c}
\end{frame}

\begin{frame}{Объект и предмет исследования}
\phantomsection\label{ux43eux431ux44aux435ux43aux442-ux438-ux43fux440ux435ux434ux43cux435ux442-ux438ux441ux441ux43bux435ux434ux43eux432ux430ux43dux438ux44f}
Julia, DrWatson, git, git flow, quatro, NodeJS
\end{frame}

\begin{frame}{Цели и задачи}
\phantomsection\label{ux446ux435ux43bux438-ux438-ux437ux430ux434ux430ux447ux438}
\begin{itemize}[<+->]
\tightlist
\item
  Установить и настроить стенд для дальнейших лабораторных работ
\end{itemize}
\end{frame}

\begin{frame}[fragile]{Материалы и методы}
\phantomsection\label{ux43cux430ux442ux435ux440ux438ux430ux43bux44b-ux438-ux43cux435ux442ux43eux434ux44b}
\begin{itemize}[<+->]
\tightlist
\item
  Процессор \texttt{pandoc} для входного формата Markdown
\item
  Результирующие форматы

  \begin{itemize}[<+->]
  \tightlist
  \item
    \texttt{pdf}
  \item
    \texttt{html}
  \end{itemize}
\item
  Автоматизация процесса создания: \texttt{Makefile}
\end{itemize}
\end{frame}

\begin{frame}{Создание репозитория}
\phantomsection\label{ux441ux43eux437ux434ux430ux43dux438ux435-ux440ux435ux43fux43eux437ux438ux442ux43eux440ux438ux44f}
\includegraphics[width=0.7\linewidth,height=\textheight,keepaspectratio]{image/1.png}

\includegraphics[width=0.7\linewidth,height=\textheight,keepaspectratio]{image/2.png}

\includegraphics[width=0.7\linewidth,height=\textheight,keepaspectratio]{image/3.png}

\includegraphics[width=0.7\linewidth,height=\textheight,keepaspectratio]{image/4.png}

\includegraphics[width=0.7\linewidth,height=\textheight,keepaspectratio]{image/5.png}

\includegraphics[width=0.7\linewidth,height=\textheight,keepaspectratio]{image/6.png}
\end{frame}

\begin{frame}{Настройка Git}
\phantomsection\label{ux43dux430ux441ux442ux440ux43eux439ux43aux430-git}
\includegraphics[width=0.7\linewidth,height=\textheight,keepaspectratio]{image/7.png}

\includegraphics[width=0.7\linewidth,height=\textheight,keepaspectratio]{image/8.png}

\includegraphics[width=0.7\linewidth,height=\textheight,keepaspectratio]{image/9.png}

\includegraphics[width=0.7\linewidth,height=\textheight,keepaspectratio]{image/10.png}

\includegraphics[width=0.7\linewidth,height=\textheight,keepaspectratio]{image/12.png}

\includegraphics[width=0.7\linewidth,height=\textheight,keepaspectratio]{image/13.png}

\includegraphics[width=0.7\linewidth,height=\textheight,keepaspectratio]{image/11.png}
\end{frame}

\begin{frame}{Настройка Git flow и проверика наличие Julia и Quatro}
\phantomsection\label{ux43dux430ux441ux442ux440ux43eux439ux43aux430-git-flow-ux438-ux43fux440ux43eux432ux435ux440ux438ux43aux430-ux43dux430ux43bux438ux447ux438ux435-julia-ux438-quatro}
\includegraphics[width=0.7\linewidth,height=\textheight,keepaspectratio]{image/14.png}

\includegraphics[width=0.7\linewidth,height=\textheight,keepaspectratio]{image/15.png}

\includegraphics[width=0.7\linewidth,height=\textheight,keepaspectratio]{image/16.png}

\includegraphics[width=0.7\linewidth,height=\textheight,keepaspectratio]{image/17.png}

\includegraphics[width=0.7\linewidth,height=\textheight,keepaspectratio]{image/18.png}

\includegraphics[width=0.7\linewidth,height=\textheight,keepaspectratio]{image/19.png}
\end{frame}

\begin{frame}{Инициализация проекта DrWatson}
\phantomsection\label{ux438ux43dux438ux446ux438ux430ux43bux438ux437ux430ux446ux438ux44f-ux43fux440ux43eux435ux43aux442ux430-drwatson}
\includegraphics[width=0.7\linewidth,height=\textheight,keepaspectratio]{image/22.png}

\includegraphics[width=0.7\linewidth,height=\textheight,keepaspectratio]{image/25.png}

\includegraphics[width=0.7\linewidth,height=\textheight,keepaspectratio]{image/20.png}

\includegraphics[width=0.7\linewidth,height=\textheight,keepaspectratio]{image/21.png}

\includegraphics[width=0.7\linewidth,height=\textheight,keepaspectratio]{image/23.png}

\includegraphics[width=0.7\linewidth,height=\textheight,keepaspectratio]{image/24.png}
\end{frame}

\begin{frame}{Написание скриптов}
\phantomsection\label{ux43dux430ux43fux438ux441ux430ux43dux438ux435-ux441ux43aux440ux438ux43fux442ux43eux432}
\includegraphics[width=0.7\linewidth,height=\textheight,keepaspectratio]{image/26.png}

\includegraphics[width=0.7\linewidth,height=\textheight,keepaspectratio]{image/27.png}

\includegraphics[width=0.7\linewidth,height=\textheight,keepaspectratio]{image/28.png}

\includegraphics[width=0.7\linewidth,height=\textheight,keepaspectratio]{image/30.png}

\includegraphics[width=0.7\linewidth,height=\textheight,keepaspectratio]{image/29.png}

\includegraphics[width=0.7\linewidth,height=\textheight,keepaspectratio]{image/31.png}

\includegraphics[width=0.7\linewidth,height=\textheight,keepaspectratio]{image/32.png}

\includegraphics[width=0.7\linewidth,height=\textheight,keepaspectratio]{image/33.png}

\includegraphics[width=0.7\linewidth,height=\textheight,keepaspectratio]{image/34.png}

\includegraphics[width=0.7\linewidth,height=\textheight,keepaspectratio]{image/35.png}

\includegraphics[width=0.7\linewidth,height=\textheight,keepaspectratio]{image/36.png}

\includegraphics[width=0.7\linewidth,height=\textheight,keepaspectratio]{image/37.png}

\includegraphics[width=0.7\linewidth,height=\textheight,keepaspectratio]{image/38.png}

\includegraphics[width=0.7\linewidth,height=\textheight,keepaspectratio]{image/39.png}
\end{frame}
\end{column}
\end{columns}
\end{frame}




\end{document}
